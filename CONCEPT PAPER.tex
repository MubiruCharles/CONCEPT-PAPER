\documentclass[12pt]{report}


% Title Page
\title{CYBER SECURITY VISUALISATION}
\author{GROUP WORK}
\date{2017-04-18}

\begin{document}
\maketitle

\section{Introduction}
\paragraph{Cyber security is a central concern for all organizations. The iterative approach to building and maintaining IT networks has left them with in-built and growing vulnerabilities. Criminals, terrorists, nation states, as well as activists and opportunistic amateurs, pose a real and persistent threat to corporate and government IT systems- a situation exacerbated by their complexity. [1] 
The fields of information visualization and visual analytics strive to leverage the unique perceptual capabilities of humans in concert with algorithmic support in order to better understand complex data. In recent years, visualization has emerged as a promising technique to better equip analysts to operate effectively in an evolving digital threat landscape. [2]
}

\section{Background}
\paragraph{Cyber security is a data-led battle, with terabytes of different information collated from around the enterprise into centralized dashboards, sometimes in dedicated network operations centers (NOCs) or security operations centers (SOCs). Existing tools do an excellent job of collating this data, often with automated real time alerts. Too frequently, however they lack the visualization capability required for humans to interpret the data. This means alerts are not effectively investigated and post- attack forensics are inefficiently managed. Integration of network visualization into the enterprise cyber security dashboard is a logical and effective approach to the manual investigation of real-time alerts and post-attack forensics.[3] }

\section{problem statement}
Often times malware on the networks of various enterprises have continuously damaged and also led to the loss of data. By carrying out cyber security visualization on the networks the malware attacks will become of no significance to the enterprise’s network data. This will provide a mode of understanding the propagation of malware and the extent of damage to the network.
\paragraph{Objectives}
General objective
•	To provide guidelines on information security data visualization and insights with repeatable processes on visualizing information security data.
Specific objectives
•	To proactively maintain a secure network perimeter.
•	To constantly look out for new emerging threats.
•	To effectively perform forensics on previous attacks.



\end{document}          

